\documentclass[letterpaper]{report}
%\usepackage[utf8]{inputenc}
\usepackage[T1]{fontenc}
\usepackage{RJournal}
\usepackage{amsmath,amssymb,array}
\usepackage{booktabs}

%% load any required packages here

\usepackage[spanish]{babel}
\usepackage{graphicx}

\hypersetup{pdftitle={TuRismo en Argentina},
            pdfkeywords={Estadísticas
Públicas; Turismo; Argentina; Software libre; Código abierto}}



\hypersetup{pdfauthor=Anónimo}

\newlength{\cslhangindent}
\setlength{\cslhangindent}{1.5em}
\newenvironment{CSLReferences}%
{\setlength{\parindent}{0pt}%
\everypar{\setlength{\hangindent}{\cslhangindent}}\ignorespaces}%
{\par}
%\usepackage[hidelinks]{hyperref}

\urlstyle{same}  % don't use monospace font for urls
\usepackage{color}
\usepackage{fancyvrb}
\newcommand{\VerbBar}{|}
\newcommand{\VERB}{\Verb[commandchars=\\\{\}]}
\DefineVerbatimEnvironment{Highlighting}{Verbatim}{commandchars=\\\{\}} 
% Add ',fontsize=\small' for more characters per line
\usepackage{framed}
\definecolor{shadecolor}{RGB}{248,248,248}
\newenvironment{Shaded}{\begin{snugshade}}{\end{snugshade}}
\newcommand{\AlertTok}[1]{\textcolor[rgb]{0.94,0.16,0.16}{#1}}
\newcommand{\AnnotationTok}[1]{\textcolor[rgb]{0.56,0.35,0.01}{\textbf{\textit{#1}}}}
\newcommand{\AttributeTok}[1]{\textcolor[rgb]{0.77,0.63,0.00}{#1}}
\newcommand{\BaseNTok}[1]{\textcolor[rgb]{0.00,0.00,0.81}{#1}}
\newcommand{\BuiltInTok}[1]{#1}
\newcommand{\CharTok}[1]{\textcolor[rgb]{0.31,0.60,0.02}{#1}}
\newcommand{\CommentTok}[1]{\textcolor[rgb]{0.56,0.35,0.01}{\textit{#1}}}
\newcommand{\CommentVarTok}[1]{\textcolor[rgb]{0.56,0.35,0.01}{\textbf{\textit{#1}}}}
\newcommand{\ConstantTok}[1]{\textcolor[rgb]{0.00,0.00,0.00}{#1}}
\newcommand{\ControlFlowTok}[1]{\textcolor[rgb]{0.13,0.29,0.53}{\textbf{#1}}}
\newcommand{\DataTypeTok}[1]{\textcolor[rgb]{0.13,0.29,0.53}{#1}}
\newcommand{\DecValTok}[1]{\textcolor[rgb]{0.00,0.00,0.81}{#1}}
\newcommand{\DocumentationTok}[1]{\textcolor[rgb]{0.56,0.35,0.01}{\textbf{\textit{#1}}}}
\newcommand{\ErrorTok}[1]{\textcolor[rgb]{0.64,0.00,0.00}{\textbf{#1}}}
\newcommand{\ExtensionTok}[1]{#1}
\newcommand{\FloatTok}[1]{\textcolor[rgb]{0.00,0.00,0.81}{#1}}
\newcommand{\FunctionTok}[1]{\textcolor[rgb]{0.00,0.00,0.00}{#1}}
\newcommand{\ImportTok}[1]{#1}
\newcommand{\InformationTok}[1]{\textcolor[rgb]{0.56,0.35,0.01}{\textbf{\textit{#1}}}}
\newcommand{\KeywordTok}[1]{\textcolor[rgb]{0.13,0.29,0.53}{\textbf{#1}}}
\newcommand{\NormalTok}[1]{#1}
\newcommand{\OperatorTok}[1]{\textcolor[rgb]{0.81,0.36,0.00}{\textbf{#1}}}
\newcommand{\OtherTok}[1]{\textcolor[rgb]{0.56,0.35,0.01}{#1}}
\newcommand{\PreprocessorTok}[1]{\textcolor[rgb]{0.56,0.35,0.01}{\textit{#1}}}
\newcommand{\RegionMarkerTok}[1]{#1}
\newcommand{\SpecialCharTok}[1]{\textcolor[rgb]{0.00,0.00,0.00}{#1}}
\newcommand{\SpecialStringTok}[1]{\textcolor[rgb]{0.31,0.60,0.02}{#1}}
\newcommand{\StringTok}[1]{\textcolor[rgb]{0.31,0.60,0.02}{#1}}
\newcommand{\VariableTok}[1]{\textcolor[rgb]{0.00,0.00,0.00}{#1}}
\newcommand{\VerbatimStringTok}[1]{\textcolor[rgb]{0.31,0.60,0.02}{#1}}
\newcommand{\WarningTok}[1]{\textcolor[rgb]{0.56,0.35,0.01}{\textbf{\textit{#1}}}}

\providecommand{\keywords}[1]{\noindent\textbf{Palabras clave:} #1}
\providecommand{\tightlist}{%
\setlength{\itemsep}{0pt}\setlength{\parskip}{0pt}}



\begin{document}

%% do not edit, for illustration only
\sectionhead{TuRismo en Argentina}
\year{2022}

\begin{article}

\title{TuRismo en Argentina}


\author{Anónimo}

\maketitle

\abstract{
Una experiencia de cómo implementamos flujos de trabajo reproducibles y
abiertos en un equipo que produce, consolida y analiza estadísticas
públicas.
}

\keywords{ Estadísticas Públicas  -  Turismo  -  Argentina  -  Software
libre  -  Código abierto }

\hypertarget{las-estaduxedsticas-de-turismo-en-la-argentina}{%
\subsection{Las estadísticas de Turismo en la
Argentina}\label{las-estaduxedsticas-de-turismo-en-la-argentina}}

La Dirección Nacional de Mercados y Estadísticas (DNMyE) de la
Subsecretaría de Desarrollo Estratégico del Ministerio de Turismo y
Deportes de la Nación (MTyD), es la responsable de las estadísticas de
turismo en el marco del sistema estadístico nacional. Como tal, entre
otras actividades, lleva adelante:

\begin{enumerate}
\def\labelenumi{(\roman{enumi})}
\item
  la coordinación de operativos estadísticos destinados a la producción
  de indicadores para la caracterización del turismo\footnote{Los mismos
    son la \textbf{\emph{Encuesta de Ocupación Hotelera (EOH)}} y la
    \textbf{\emph{Encuesta de Turismo Internacional (ETI)}}, junto al
    Instituto Nacional de Estadísticas y Censos (INDEC); y, por cuenta
    propia, la \textbf{\emph{Encuesta de Viaje y Turismo de los Hogares
    (EVyTH).}}};
\item
  la armonización, explotación y análisis de una variedad de registros
  administrativos\footnote{Ejemplos de ellos son (a) algunos producidos
    en la orbita del MTyD - como registro de Agencias de Viajes; u
    (b)otros gestionados y cedidos por organismos como la Dirección
    Nacional de Migraciones (DNM - para la estimación del turismo
    internacional por todas las vías); la Agencia Nacional de
    Aeronavegación Comercial (ANAC - sobre oferta y demanda de vuelos,
    tanto internacional como de cabotaje); la Comisión Nacional
    Reguladora de Transporte (CNRT - con datos de registro de servicios
    de transporte terrerstre), o la Administración de Parques Nacionales
    (APN - con registros de visitaciones a áreas protegidas naturales de
    adminsitración nacional).}.
\end{enumerate}

Este conjunto de tareas requiere de una serie de procesos que implican
la captura, limpieza, consistencia, consolidación, análisis y
comunicación de datos que provienen de múltiples fuentes, con diversos
formatos y estructuras.

\hypertarget{un-nuevo-flujo-de-trabajo}{%
\subsection{Un nuevo flujo de trabajo}\label{un-nuevo-flujo-de-trabajo}}

De un tiempo a esta parte, la DNMyE viene encarando un proyecto de
transformación de los procesos detrás de las acciones antes reseñadas.
Con un diagnóstico inicial de que se podía avanzar en la implementación
de un flujo de trabajo más sistemático y reproducible; que implicara la
apertura de fuentes de datos y de procesos; y que pudiera ser compartido
en diversos soportes con usuarios potenciales diversos, se decidió
avanzar en un plan de trabajo para lograr esos objetivos.

\textbf{Primeros pasos}:

\begin{itemize}
\item
  el punto de partida consistió en armar un \emph{stack} de trabajo,
  donde \textbf{R} juega un rol central;
\item
  se creo una instancia de capacitación interna con el objetivo de
  acercar al equipo de trabajo de la dirección (muches con experiencia
  de análisis estadístico, otres con práctica en programación en
  diferentes \emph{softwares} -en su mayoría propietarios-) el uso de
  estas nuevas herramientas y bajar la curva de aprendizaje\footnote{En
    el contexto del trabajo remoto, a causa de las restricciones a la
    movilidad por la pandemia del Covid19, esta organización nos
    permitió abordar problemas de administración de bases de datos y el
    mantenimiento de librerías, por ejemplo, que aseguraran la
    reproducibilidad del trabajo.}. Esta capacitación inicial devino en
  encuentros semanales dedicados a compartir problemáticas específicas y
  avanzar sobre diferentes temáticas en cuanto el procesamiento de datos
  con \textbf{R};
\item
  se procuró la creación y configuración de un servidor dedicado para la
  DNMyE donde corre un
  \textbf{\href{https://www.rstudio.com/products/rstudio/download-server/}{RStudio
  Server}}. En él, cada usuarie cuenta con un acceso para poder hacer
  uso del \emph{software} desde un navegador\footnote{Este espacio de
    encuentro interno se formalizó luego también en un esquema de
    capacitaciones destinado a agentes que cumplen funciones técnicas en
    las oficinas encargadas de las estadíticas de turismo, no solamente
    del gobierno nacional, sino también de gobiernos provinciales y
    locales. El material (\textbf{\emph{bookdown}} y videos de las
    capacitaciones se encuentran alojadas en
    \textbf{\href{https://armonizacion.yvera.tur.ar/}{armonziacion.yvera.tur.ar}}).}.
  El mismo permite, además, desplegar aplicaciones (\emph{Shiny)} y un
  sistema de archivos que permite a todes hacer uso de las fuentes de
  datos necesarias para cada proceso;
\item
  se creo un repositorio de organización (\emph{de desarrollo}) en
  \emph{GitHub}, el cual permitió acumular la producción de código y
  trabajar con un sistema de control de versiones. Con el paso del
  tiempo se creo un segundo repositorio (\emph{de producción}) donde
  paulatinamente se van publicando aplicaciones \emph{Shiny}, libros
  digitales y reportes, entre otros:
  \textbf{\url{http://github.com/dnmye-minturdep/}}.
\end{itemize}

El proceso no fue lineal, se fueron incorporando herramientas y
tecnologías en la medida que se generaban nuevos productos y demandas.
Paulatinamente, la generación de informes, tradicionalmente producidos
en procesadores de texto o planillas de cálculo, comenzó a desarrollarse
con código fuente. Con el tiempo estos informes se fueron convirtiendo
en reportes interactivos y luego derivando en tableros como soporte
alternativo para la consulta de información. La generación de estas
nuevas formas de acceder a datos y estadísticas producidas por la DNMyE
nos llevó a organizar de un nuevo modo la comunicación de estos
productos.

\begin{center}\includegraphics[width=0.5\linewidth]{sinta} \end{center}

\hypertarget{comunicaciuxf3n}{%
\subsection{COMUNICACIÓN}\label{comunicaciuxf3n}}

El \href{https://www.yvera.tur.ar/sinta/}{\textbf{Sistema de Información
Turística de la Argentina (SINTA)}} es el resultado de esta experiencia.
Un portal único, en constante actualización, que ofrece una
multiplicidad de formas de acceso a la información y datos producidos
por la DNMyE, muchos de ellos trabajados con \textbf{R} de punta a
punta. Destacamos acá:

\begin{enumerate}
\def\labelenumi{\arabic{enumi}.}
\item
  el procesamiento de los datos publicados en el \textbf{Portal de Datos
  Abiertos (\href{http://datos.yvera.gob.ar/}{datos.yvera.tur.ar})} que
  se administra con el paquete \CRANpkg{ckanr} (Chamberlain et al.
  2021).
\item
  la producción de \textbf{Documentos Técnicos} gracias al soporte de
  \CRANpkg{bookdown} (Xie 2016), en el marco del programa de
  \textbf{ARMONIZACIÓN} de las estadísticas de turismo en las provincias
  y la generación de \textbf{INFORMES
  (\href{https://www.yvera.tur.ar/sinta/informe}{yvera.tur.ar/informes})}
  a partir de \emph{templates} generados con \CRANpkg{pagedown} (Xie et
  al. 2021).
\item
  una serie de micrositios generados con \CRANpkg{distill} (Allaire et
  al. 2021).
\end{enumerate}

\begin{itemize}
\item
  el \emph{blog} \textbf{BITÁCORA
  (\href{https://bitacora.yvera.tur.ar/}{bitacora.yvera.tur.ar})};
\item
  micrositios de estadísticas desde una perspectiva federal: el proyecto
  de \textbf{ARMONIZACION} antes citado
  (\href{https://armonizacion.yvera.tur.ar/}{armonizacion.yvera.tur.ar})y
  un \textbf{MONITOR} con indicadores agrupados a nivel provincial
  (\href{https://provincias.yvera.tur.ar/}{provincias.yvera.tur.ar}).
\item
  un micrositio para la consulta interactiva de últimos datos -
  \textbf{REPORTES}- e información desagregada en -\textbf{TABLEROS}
  (\href{https://tableros.yvera.tur.ar/}{tableros.yvera.tur.ar}).
\item
  entre los desarrollos de aplicaciones con \CRANpkg{shiny} (Chang et
  al. 2021) se destacan los tableros de consulta de datos de
  \textbf{(i)} turismo internacional, \textbf{(ii)} del padrón único
  nacional de alojamientos y de \textbf{(iii)} conectividad
  aerocomercial - de pronta publicación.
\end{itemize}

\textbf{HERRAMIENTAS}

Por último, con la acumulación de experiencias, empezamos a sistematizar
buena parte del trabajo en herramientas que nos permiten un flujo de
trabajo más eficiente, buscando reducir en lo posible la repetición de
tareas o procesos reduciendo la posibilidad de errores, automatizando
flujos de trabajo y permitiendo el control de cambios para administrar
la evolución de los proyectos. En este sentido es que avanzamos con el
desarrollo de \textbf{paquetes en R} que nos facilitan los pasos típicos
en nuestro flujo usual de trabajo:

\begin{itemize}
\item
  Con
  \href{https://github.com/dnme-minturdep/herramientas}{\textbf{\texttt{\{herramientas\}}}}
  (Tiscornia et al. 2022) buscamos simplificar tareas cotidianas de
  acceso a datos en los servidores, limpieza de determinada clases de
  variables o cálculos usuales de indicadores.
\item
  Con
  \href{https://github.com/dnme-minturdep/comunicacion}{\textbf{\texttt{\{comunicacion\}}}}
  (Tiscornia and Ruiz Nicolini 2022), por otro lado, podemos generar
  informes, reportes o visualizaciones que utilizan los parámetros y
  estilos de la comunicación oficial.
\item
  Por último vale mencionar a \emph{\textbf{mapeAr}}
  (\href{https://tableros.yvera.tur.ar/mapeAr}{tableros.yvera.tur.ar/mapeAr}).
  Una herramienta creada con \emph{Shiny}, también presente en esta
  convocatoria, que permite la generación de visualizaciones
  georeferenciadas a través de una interfaz, sin necesidad de tener
  conocimientos de programación o de \textbf{R} en particular.
\end{itemize}

\hypertarget{referencias}{%
\section*{Referencias}\label{referencias}}
\addcontentsline{toc}{section}{Referencias}

\hypertarget{refs}{}
\begin{CSLReferences}{1}{0}
\leavevmode\vadjust pre{\hypertarget{ref-distill}{}}%
Allaire, JJ, Rich Iannone, Alison Presmanes Hill, and Yihui Xie. 2021.
\emph{Distill: 'R Markdown' Format for Scientific and Technical
Writing}. \url{https://CRAN.R-project.org/package=distill}.

\leavevmode\vadjust pre{\hypertarget{ref-ckanr}{}}%
Chamberlain, Scott, Imanuel Costigan, Wush Wu, Florian Mayer, and Sharla
Gelfand. 2021. \emph{Ckanr: Client for the Comprehensive Knowledge
Archive Network ('CKAN') API}.
\url{https://CRAN.R-project.org/package=ckanr}.

\leavevmode\vadjust pre{\hypertarget{ref-shiny}{}}%
Chang, Winston, Joe Cheng, JJ Allaire, Carson Sievert, Barret Schloerke,
Yihui Xie, Jeff Allen, Jonathan McPherson, Alan Dipert, and Barbara
Borges. 2021. \emph{Shiny: Web Application Framework for r}.
\url{https://CRAN.R-project.org/package=shiny}.

\leavevmode\vadjust pre{\hypertarget{ref-herramientas}{}}%
Tiscornia, Pablo, Juan Juara, Juan Urricariet, and Juan Pablo Ruiz
Nicolini. 2022. \emph{Herramientas: Conjunto de Funciones Para El
Procesamiento de Datos En La DNMyE}.
\url{https://dnme-minturdep.github.io/herramientas/}.

\leavevmode\vadjust pre{\hypertarget{ref-comunicacion}{}}%
Tiscornia, Pablo, and Juan Pablo Ruiz Nicolini. 2022.
\emph{Comunicacion: DNMyE - Comunicación}.
\url{https://d4t4tur.github.io/comunicacion}.

\leavevmode\vadjust pre{\hypertarget{ref-bookdown}{}}%
Xie, Yihui. 2016. \emph{Bookdown: Authoring Books and Technical
Documents with {R} Markdown}. Boca Raton, Florida: Chapman; Hall/CRC.
\url{https://bookdown.org/yihui/bookdown}.

\leavevmode\vadjust pre{\hypertarget{ref-pagedown}{}}%
Xie, Yihui, Romain Lesur, Brent Thorne, and Xianying Tan. 2021.
\emph{Pagedown: Paginate the HTML Output of r Markdown with CSS for
Print}. \url{https://CRAN.R-project.org/package=pagedown}.

\end{CSLReferences}




\end{article}
\end{document}
